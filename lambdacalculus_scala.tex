\documentclass[10pt]{article}

\usepackage{latexsym}
\usepackage{amssymb}
\usepackage{code}
\usepackage{graphicx}

\usepackage{listings}



\usepackage[english]{babel}
%\usepackage{blindtext}
%\usepackage{fontspec}
%\setmainfont{Arial}

%\renewcommand{\familydefault}{\sfdefault}
%\usepackage{blindtext}
\pagenumbering{arabic}  % Arabic page numbers GM July 2000

\usepackage{authoraftertitle}
\usepackage{fancyhdr}
% Clear the header and footer
\fancyhead{}
\fancyfoot{}
%\lhead{\includegraphics[height=0.7cm]{../logo/nyp-logo-int.png} }
\rhead{\scriptsize \MyTitle}
%\lfoot{\scriptsize Specialist Diploma in Business \& Big Data Analytics\\ Copyright
%  \copyright\ Nanyang Polytechnic. All Rights Reserved.}
\rfoot{\thepage}
\pagestyle{fancy}


% Martin's macros
   

% groups

%%\newtheorem{lemma}{Lemma}
%%\newtheorem{definition}{Definition}
%%\newtheorem{theorem}{Theorem}
%%\newtheorem{corollary}{Corollary}
%%\newtheorem{remark}{Remark}
%%\newtheorem{conjecture}{Conjecture}
%%\newtheorem{example}{Example}

% arrays

\newcommand{\bi}{\begin{array}[t]{@{}l@{}}}
\newcommand{\ei}{\end{array}}
\newcommand{\ba}{\begin{array}}
\newcommand{\ea}{\end{array}}
\newcommand{\bda}{\[\ba}
\newcommand{\eda}{\ea\]}
\newcommand{\bp}{\begin{quote}\tt\begin{tabbing}}
\newcommand{\ep}{\end{tabbing}\end{quote}}
\newcommand{\set}[1]{\left\{
    \begin{array}{l}#1
    \end{array}
  \right\}}
\newcommand{\sset}[2]{\left\{~#1 \left|
      \begin{array}{l}#2\end{array}
    \right.     \right\}}


% symbols

\newcommand{\good}{good}
 
\newcommand{\err}{{\bf W}}
\newcommand{\VV}{{\cal V}} 
\newcommand{\GT}{{\cal T}_G}
\newcommand{\KK}{{\cal K}}
\newcommand{\TT}{{\cal T}}
\newcommand{\TC}{\mbox{\it TC}}
\newcommand{\UL}{\mbox{UL}}
\newcommand{\sUL}{\mbox{\scriptsize UL}}
\newcommand{\true}{\mbox{\sf true}}
\newcommand{\Int}{\mbox{\it Int}}
\newcommand{\Bool}{\mbox{\it Bool}}
\newcommand{\SF}{\mbox{$\cal S$}}
\newcommand{\tauvec}{\bar{\tau}}
\newcommand{\tvec}{\bar{t}}
\newcommand{\kvec}{\bar{k}}
\newcommand{\xvec}{\bar{x}}
\newcommand{\muvec}{\bar{\mu}}
\newcommand{\alphavec}{\bar{\alpha}}
\newcommand{\betavec}{\bar{\beta}}
\newcommand{\gammavec}{\bar{\gamma}}
\newcommand{\thetavec}{\bar{\theta}}
\newcommand{\deltavec}{\bar{\delta}}
\newcommand{\bvec}{\bar{b}}
\newcommand{\typo}{\Gamma}
\newcommand{\tenv}{\Gamma}
\newcommand{\typoinit}{\typo_0}
\newcommand{\static}{S}
\newcommand{\dynamic}{D}
\newcommand{\sstatic}{{\scriptstyle S}}
\newcommand{\sdynamic}{{\scriptstyle D}}
\newcommand{\BTA}{\mbox{BTA}}
\newcommand{\baseshape}{\bot}
\newcommand{\TCT}{P}
\newcommand{\Haskell}{H98}
\newcommand{\TCTinit}{P_o}

% figures, rules


\newcommand{\tlabel}[1]{\mbox{(#1)}}
\newcommand{\fig}[3]
        {\begin{figure*}[t]#3\
        \caption{\label{#1}#2}\ \hrulefill\ \end{figure*}}
\newcommand{\figurebox}[1]
        {\fbox{\begin{minipage}{\textwidth} #1 \end{minipage}}}
\newcommand{\boxfig}[3]
        {\begin{figure*}\figurebox{#3\caption{\label{#1}#2}}\end{figure*}}
\newcommand{\myirule}[2]{{\renewcommand{\arraystretch}{1.2}\ba{c} #1
                      \\ \hline #2 \ea}}

% relations, operators

\newcommand{\sat}{\mbox{\it sat}}
\newcommand{\entail}{\mbox{\it entail}}
\newcommand{\unambig}{\mbox{\it unambig}}
\newcommand{\complete}{\mbox{\it complete}}

\newcommand{\projexclusion}[1]{\mbox{$\bar{\exists}#1$}}
\newcommand{\funinst}{\mbox{\it inst}}
%%\renewcommand{\inst}{\, \vdash^{\scriptstyle i} \,}
%%\newcommand{\inst}{\, \vdash \,}
\newcommand{\ileq}{\preceq}
\newcommand{\ieq}{\simeq}
\newcommand{\topannot}[1]{|#1|}
\newcommand{\gendelta}{\mbox{\it gen}_{\delta}}
\newcommand{\genbeta}{\mbox{\it gen}_{\beta}}
\newcommand{\gen}{\mbox{\it gen}}
\newcommand{\normalize}{\mbox{\it normalize}}
\newcommand{\fail}{\mbox{\it fail}}
\newcommand{\fixpt}{\mbox{${\cal F}$}}
\newcommand{\testpa}{\mbox{${\cal T}$}}
\newcommand{\addpa}{\mbox{${\cal A}$}}
\newcommand{\matchpa}{\mbox{${\cal M}$}}
\newcommand{\refmatchpa}{\mbox{${\cal R}$}}
\newcommand{\wft}[1]{\mbox{\it wft}(#1)}
\newcommand{\tv}{\mbox{\it fv}}
\newcommand{\stv}{\mbox{\it semfv}}
\newcommand{\range}{\mbox{\it range}}
\newcommand{\lub}{\wedge}
\newcommand{\tyrel}[2]{(#1 \preceq #2)}
\newcommand{\eqty}[2]{#1=#2}
%%\newcommand{\ts}{\, \vdash \,}
\newcommand{\turns}{\, \vdash \,}
\newcommand{\tsinst}{\, \vdash^i \,}
\newcommand{\tsttv}{\, \vdash^{ttv} \,}
\newcommand{\tsUL}{\, \vdash_{\mbox{\scriptsize UL}} \,}
\newcommand{\trc}{\, \vdash_{\scriptstyle \id{inf}} \,}
\newcommand{\trcp}{\, \vdash_{\scriptstyle inf}' \,}
\newcommand{\genrel}{\, \vdash^{\scriptstyle gen} \,}
\newcommand{\asubty}[2]{(#1 \leq_a #2)}
\newcommand{\ssubty}[2]{(#1 \leq_s #2)}
\newcommand{\fsubty}[2]{(#1 \leq_f #2)}
\newcommand{\subty}[2]{(#1 \leq #2)}
\newcommand{\doubleb}[1]{[\![ #1 ]\!]}
\newcommand{\id}[1]{\mbox{\it #1}}


\newcommand{\mynote}[1]{$\spadesuit${\bf #1}$\clubsuit$}

% spacing

\newcommand{\sgap}{\quad}
\newcommand{\bgap}{\quad\quad}

% reserved words

\newcommand{\mathem}{\sf}
\newcommand{\IN}{\mbox{\mathem in}}
\newcommand{\LET}{\mbox{\mathem let}}
\newcommand{\MLET}{\mbox{\mathem mlet}}
\newcommand{\LETREC}{\mbox{\mathem letrec}}
\newcommand{\FIX}{\mbox{\mathem fix}}
\newcommand{\CASE}{\mbox{\mathem case}}
\newcommand{\TYCASE}{\mbox{\mathem typecase}}
\newcommand{\ELSE}{\mbox{\mathem else}}
\newcommand{\IF}{\mbox{\mathem if}}
\newcommand{\OF}{\mbox{\mathem of}}
\newcommand{\THEN}{\mbox{\mathem then}}
\newcommand{\INST}{\mbox{\mathem instance}}
\newcommand{\OVER}{\mbox{\mathem overload}}
\newcommand{\CLASS}{\mbox{\mathem class}}
\newcommand{\WHERE}{\mbox{\mathem where}}


%%% Local Variables: 
%%% mode: latex
%%% TeX-master: "bta"
%%% End: 

\renewcommand{\topfraction}{0.95}
\renewcommand{\textfraction}{0.02}
\renewcommand{\floatpagefraction}{0.95}
\newcommand{\diff}{-}


\bibliographystyle{plain}

%%\textwidth      150mm \textheight     210mm \oddsidemargin  -2mm \evensidemargin -2mm

\renewcommand{\baselinestretch}{0.987}

\setlength{\parskip}{0.0in}


\newcommand{\Nturns}{\, \vdash_{\mbox{\scriptsize lnf}} \,}

\newcommand{\tr}[1]{}
%%\newcommand{\tr}[1]{#1}
%%\newcommand{\nottr}[1]{#1}
\newcommand{\nottr}[1]{}

%\newcommand{\implies}{\supset}
\newcommand{\clabel}[1]{\mbox{(#1)}}
\newcommand{\rat}[1]{\rightarrowtail_{#1}}
\newcommand{\arr}{\rightarrow}
\newcommand{\arrow}{\rightarrow}
\newcommand{\Arr}{\Rightarrow}
\newcommand{\atsign}{@}
\newcommand{\simparrow}[0]{\Longleftrightarrow}
\newcommand{\proparrow}[0]{\Longrightarrow}
\newcommand{\comment}[1]{}
\newcommand{\ignore}[1]{}
%\newcommand{\kl}[1]{{\bf KL:#1}} %%%{\marginpar{\sc kl}{\bf #1}}
\newcommand{\kl}[1]{}
\newcommand{\ms}[1]{{\bf MS:#1}}       %%{\marginpar{\sc ms}{\bf #1}}
%%\newcommand{\jw}[1]{\marginpar{\sc jw}{\bf #1}}
%% \newcommand{\answer}[1]{#1}

\newcommand{\pjs}[1]{}
%%\newcommand{\ms}[1]{}
\newcommand{\jw}[1]{}
%%\newcommand{\kl}[1]{}


\newcommand{\pow}{\^{}}
\newcommand{\venv}{\Delta}
\newcommand{\mleq}{\mbox{\tt leq}}
\newcommand{\mmleq}{\leq}
\newcommand{\mas}{\mbox{\tt as}}
\newcommand{\mfix}{\mu}

\newcommand{\mysection}[1]{\vspace*{-2mm}\section{#1}\vspace*{-1mm}}
\newcommand{\mysubsection}[1]{\vspace*{-1mm}\subsection{#1}\vspace*{-1mm}}

\newcounter{cnt}
\newtheorem{ex}{Example}
%\newenvironment{example}{
%        \begin{ex}\rm}%
%    {\hfill$\Box$\end{ex}}

\newenvironment{nexample}{
        \begin{ex}\rm}%
        {\end{ex}}

%%\newtheorem{rem}{Remark}
%%\renewenvironment{remark}{
%%        \begin{rem}\rm}%
%%    {\hfill$\Box$\end{rem}}

%%\newenvironment{myexample}{
%%        \begin{ex}\rm}%
%%  {\hfill $\Diamond$\end{ex}}
%\newenvironment{example}{
%        \begin{example}\rm}%
%    {\end{example}}

\newenvironment{ttline}{\begin{trivlist}\item \tt}{\end{trivlist}}
\newenvironment{ttprog}{\begin{trivlist}\small\item \tt
        \begin{tabbing}}{\end{tabbing}\end{trivlist}}


\newcommand{\figcode}[1]
        {\begin{figure*}[t]#1
        \end{figure*}}


\title{%ITD353 Massively Parallel Computing for Big Data \\
  Implementing Lambda Calculus in Scala }
%\author{Kenny Zhuo Ming Lu\\
%  \multicolumn{1}{p{.7\textwidth}}{\centering
%  \emph{School of Information Technology \\ Nanyang Polytechnic \\
%    180 Ang Mo Kio Avenue 8, Singapore 569830}}
%}


\begin{document}
\maketitle \makeatactive
\thispagestyle{fancy}

\lstset{language=Python}

%\begin{abstract}
%\end{abstract}


\section{Learning Outcomes} \label{sec:aims}
\begin{itemize}
 \item Understand and apply algebraic data type
 \item Understand and apply State Monad
\end{itemize}


\section{Implementating Lambda Caculus}
\begin{enumerate}
\item Recall that Lambda Calculus syntax is defined as the following EBNF

\bda{rccl}
 \tlabel{Lambda Terms} & t & ::= & x \mid \lambda x.t \mid t~t
\eda

\item Check out the source code. Download and open {\tt LambdaCaculus.scala}, we will see 
\begin{code}
import scala.collection.Set

object LambdaCalculus {

  sealed trait Term
  case class App(t1:Term, t2:Term) extends Term
  case class Lam(x:String, t:Term) extends Term
  case class Var(x:String) extends Term

  val e0 = App(Var("x"), Lam("y",Var("y")))

...
\end{code} 
%$ 
Note that it is clear that the Scala algebraic data type above has the exact mapping to our EBNF definition. {\tt e0} defines a lambda term $(x~(\lambda y.y))$.
Your first task is to define the following
  \begin{itemize}
  \item $(\lambda x.x)~(\lambda y.y)~(\lambda z.z)$
  \item $\lambda x.(\lambda x.x~x~y)~(\lambda z.z~x)$
  \item $(\lambda x.x~x)~(\lambda x.x~x)$
  \item $(\lambda y.\lambda x.y) ~(\lambda z.z)~ ((\lambda x.x~x)~(\lambda x.x~x))$    
  \end{itemize}
%
Let's call them {\tt e1}, {\tt e2}, {\tt e3} and {\tt e4}
  \item In {\tt LambdaCaculus.scala}, we also found the following definition of {\tt pretty} which pretty-prints the given lambda term. 
\begin{code}
  def pretty(t:Term):String = t match {
    case Var(x)     => x
    case Lam(v,t)   => s"(\\${v}->" ++ pretty(t) ++")"
    case App(t1,t2) => pretty(t1) ++ " " ++ pretty(t2)
  }
\end{code}
%$
where {\tt s"..."} denote a Scala string interpolation, in which {\tt \${v}} refers to the value of variable {\tt v}.
For instance, 
{\tt pretty(e0)} yields 
\begin{code}
x \y->y
\end{code}
Recall the definition of $fv$, 
\bda{rcl}
fv(x) & = & \{x\}\\
fv(\lambda x.t) & = & fv(t) \diff \{x\} \\ 
fv(t_1~t_2) & = & fv(t_1) \cup fv(t_2) 
\eda
%
your next task is to define the {\tt fv} function in Scala which should have the function signature of 
\begin{code}
   def fv(t:Term):Set[String] = Set() // TODO:CHANGE ME
\end{code}
%\item Implement An Interpreter of Lambda Calculus 
\item Thirdly, we consider the implementation of the substitution $\lbrack t/x \rbrack$, which is as follows
\begin{code}
  type Subst = (String, Term)
\end{code}
% 
Recall the application of a substitution to a lambda term is defined as following equalities.
\bda{rcll}
 \lbrack t_1 / x \rbrack x & = & t_1 \\
 \lbrack t_1 / x \rbrack y & = & y & \mbox{if}~ x \neq y \\
 \lbrack t_1 / x \rbrack (t_2~t_3) & = & \lbrack t_1 / x \rbrack t_2~
 \lbrack t_1 / x \rbrack t_3 & \\
 \lbrack t_1 / x \rbrack \lambda y.t_2 & = & \lambda y. \lbrack t_1 / x
 \rbrack t_2 & \mbox{if}~ y\neq x~ \mbox{and}~ y \not \in fv(t_1)
\eda
%
Which can be implemented as follows with minor tweaks,
\begin{code}
def subst(s:Subst, t:Term, max:Int) : (Term,Int) = s match {
    case (x,t1) => t match {
      case Var(y) if x == y => (t1,max)
      case Var(y)           => (Var(y),max)
      case App(t2,t3)       => {
        val (u2,max2) = subst(s,t2,max)
        val (u3,max3) = subst(s,t3,max2)
        (App(u2, u3),max3)
      }
      case Lam(y,t) if x == y => { // (1) 
        (Lam(y,t),max)
      }
      case Lam(y,t) if fv(t1).contains(y) => { // (2)
        val w = "x"+max
        val max1 = max + 1
        rename(w,y, t,max1) match {
          case (t2,max2) => subst(s,Lam(w,t2),max2)
        }
      }
      case Lam(y,t) => {
        subst(s,t,max) match {
          case (t1,max1) => (Lam(y,t1),max1)
        }
      }
    }
  }
\end{code}
%
The above implementation defers from the defintion at {\tt (1)} and {\tt (2)}. At {\tt (1)}, we consider the case where substitution variable {\tt x} is identical to
the lambda bound variable {\tt y}. Based our defintion, we should stop and signal an error. In the actual implementation, we just return the original lambda abstraction without substitution.
This is fine, because if we were to rename {\tt y} in the lambda abstraction, we get back the same term (under $\alpha$ renaming), since {\tt y} is identical {\tt x}, there is no other
possible free variable in {\tt t} that could be equal to {\tt x}. 
At {\tt (2)}, we found the lambda variable {\tt y} is in the set of free variables in {\tt t}, the payload of the substitution. We have to rename {\tt y} by a fresh variable {\tt w}. {\tt w} is fresh because it is
variable suffixed by a running number {\tt max}. Note that because of this requirement we have to thread {\tt max} through the {\tt subst} function, that is, 
expecting {\tt max} as its input and returning the updated {\tt max} as output.  The alpha renaming is implemented in {\tt rename}.
\begin{code}
  def rename(w:String, v:String, t:Term, max:Int):(Term,Int) = 
    subst((v,Var(w)),t,max)
\end{code}
%
{\tt w} is the fresh variable, {\tt v} is the lambda bound variable to be replaced, {\tt t} is the body. The task is to replace all free occurrances of {\tt v} in {\tt t} by {\tt w}.
We can achieve that by calling {\tt subst((v,Var(w)),t,max)}, because {\tt w} must be fresh and it must not have been used by any lambda abstraction in {\tt t}.
\item With {\tt subst} nailed, we can move on with the implementation of the reduction strategy, specifically, we conslider NOR which is easier to be implemented in Scala.
NOR favors to apply $\beta$-reduction to the outer-most left-most redex in the given lambda term. In Scala, we can use pattern matching to ``scan'' through the structure of the input lambda
term from outer-most to inner-most.
\begin{code}
  def nor(t:Term, max:Int):(Term,Int) = t match { // we apply
    case Var(x)   => (Var(x),max)
    case App(Lam(v,t1),t2) => {
      val s = (v,t2)
      subst(s,t1,max)
    }
    case App(t1,t2) => nor(t1,max) match {
      case (t3,max3) if equiv(t1,t3,max3) => nor(t2,max3) match {
        case (t4,max4) => (App(t3,t4),max4)
      }
      case (t3,max3) => (App(t3,t2),max3)
    }
    // exercise: implement the missing case for Lam(v,t1) TODO: CHANGE ME
  }

def equiv(t1:Term, t2:Term, max:Int):Boolean = true // TODO: CHANGE ME
\end{code}
%
The {\tt nor} function is defined inductively over the structore of the input lambda term {\tt t}. In case of a variable, we return the variable.
In case of an application of a redex, we apply $\beta$ reduction. In case of an application of non-redex, we reduce the left sub-term if it is reducible,
otherwise, we reduce the right sub-term.

Note that there are two missing parts in the above implementation. 

First of all the case of lambda abstraction is missing. In case of a lambda abstraction, 
we should ``look'' inside the body of the function and search for redex to reduce. Your task is to implement the missing case.

Secondly, the equivalence test {\tt equiv} among two lambda terms is missing.
We can't use {\tt ==} nor {\tt eq} direclty from Scala, because they compare the referenced memory addresses among objects. 
We need something test for structural equivalence and ignore the 
variable names, which is defined as the $t_1 \equiv t_2$ relation follows,
\bda{ccl}
x \equiv y  & \iff & x = y \\
\lambda x.t_1  \equiv  \lambda y.t_2 & \iff & \lbrack w/x \rbrack t_1 \equiv \lbrack w/y \rbrack t_2 ~ \mbox{where}~ w~ \mbox {is a fresh variable}. \\ 
(t_1~t_2) \equiv (t_3~t_4) & \iff & t_1 \equiv t_3~\wedge~t_2\equiv t_4
\eda
%
Your task is to fix the {\tt equiv} function. The hint is that you need the {\tt max} in order to generate fresh variable $w$.

\item With the above implemented, we can make a step forward to implement a ``big-step'' reduction which recursively applies {\tt nor} until the lambda term remains unchanges.
\begin{code}
  def norS(t:Term):Term = {
    def reduce(t:Term,max:Int):(Term,Int) = nor(t,max) match {
      case (t1,max1) => {
        if (equiv(t,t1)) {
          (t1,max1)
        } else {
          reduce(t1,max1)
        }
      }
    }
    reduce(t,0)._1
  }
\end{code}
Your task to test {\tt norS} against {\tt e0}, {\tt e1}, {\tt e2}, {\tt e3} and {\tt e4}.

\end{enumerate}

\section{Simplify variable threading using State Monad}

Note that in the above implementation, in order to generate the fresh variable, we've been threading the {\tt max} variable around. This works but badly affects
the readbility and the maintainability of the code. 

A state monad is a monad which provides the abstraction of all transition computations that read a state as inputs and returns the modified state as part of the output. For instance {\tt max} in our case.
Concretely, 
\begin{code}
trait Monad[F[_]] {
  def point[A](a: => A): F[A] 
  def bind[A, B](fa: F[A])(f: A => F[B]): F[B]
} 

case class State[S,A]( run : S => (S,A) ) 

implicit def stateMonad[S] = new Monad[({type L[B] = State[S,B]})#L] { // (1)
  override def point[A](a: => A): State[S, A] =  (st => (a, st))             
  override def bind[A, B](p: State[S, A])(f: A => State[S, B]): State[S, B] =
    s1 => { p.run(s1) match {
         case (s2,a) => f(a).run(s2) 
      }
   }
}
\end{code}
%
At {\tt (1)} we define a type class instance of state monad. To allow higher order type, we use 
{\tt ({type L[B] = State[S,B]})\#L} to denotes that {\tt L} is partially applied type constructor 
in which {\tt State} applied partially applied to {\tt S}, for all possible type paremeter {\tt B}.


The above is a simplified definition of state monad which shipped with scalaz. With scalaz, we can simplify 
our lambda calculus implentation with the following combinator, 
\begin{code}
  case class Ctxt(maxid:Int)

  def freshVar:State[Ctxt,String] = for {
    ctxt <- get[Ctxt] // local type inference is too weak
    val maxid = ctxt match { case Ctxt(maxid) => maxid }
    _ <- put(Ctxt(maxid+1))
  } yield ("x" +maxid)
\end{code}
%
{\tt freshVar} get the current {\tt maxid} from the state {\tt Ctxt}, increments it and returns a fresh variable. Now the implementation of {\tt subst}
is much simpler
\begin{code}
  def subst(s:Subst, t:Term): State[Ctxt,Term] = s match {
    case (x,t1) => t match {
      case Var(y) if x == y => point(t1)
      case Var(y)           => point(Var(y))
      case App(t2,t3)       => for {
        u2 <- subst(s,t2)
        u3 <- subst(s,t3)
      } yield App(u2, u3)
      case Lam(y,t) if x == y => {
        point(Lam(y,t))
      }
      case Lam(y,t) if fv(t1).contains(y) => for {
        w  <- freshVar
        t2 <- rename(w, y, t)
        r  <- subst(s,Lam(w,t2))
      } yield r
      case Lam(y,t) => for {
        t1 <- subst(s,t)
      } yield Lam(y,t1)
    }
  }

  def rename(w:String, v:String, t:Term):State[Ctxt,Term] =
    subst((v,Var(w)),t)
\end{code}

To check out the code template, unzip {\tt lambdacalculus\_monad.zip}, and open {\tt src/main/scala/examples/LambdaCalculusState.scala}.
Make the changes like what you did in the previous section (modula the part of updating the {\tt max} variable and the monad for comprehension).

To run it, you need to use go to the extracted folder, type {\tt sbt console}. 
Note that in case you are behind firewall, you need to modify {\tt sbtconfig.txt} (In Windows, {\tt C:/Program Files/sbt/config/sbtconfig.txt} with the following
\begin{code}
-Dhttp.proxyHost=proxy.hs-karlsruhe.de
-Dhttp.proxyPort=8888
-Dhttp.proxyUser=username
-Dhttp.proxyPassword=password
-Dhttps.proxyHost=proxy.hs-karlsruhe.de
-Dhttps.proxyPort=8888
-Dhttps.proxyUser=username
-Dhttps.proxyPassword=password
\end{code}
But remember to remove it before you logout.

\section{Implementing AOR}
Can you implement AOR? \\ 
 Hint, we still traverse the lambda term structurally from outer to inner, excep that  when we see a redex of shape {\tt App(Lam(v,t1),t2)} we need to check whether 
there exists inner redex in {\tt t1} or {\tt t2}.

\end{document}
