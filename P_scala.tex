\documentclass[10pt]{article}

\usepackage{latexsym}
\usepackage{amssymb}
\usepackage{code}
\usepackage{graphicx}

\usepackage{listings}



\usepackage[english]{babel}
%\usepackage{blindtext}
%\usepackage{fontspec}
%\setmainfont{Arial}

%\renewcommand{\familydefault}{\sfdefault}
%\usepackage{blindtext}
\pagenumbering{arabic}  % Arabic page numbers GM July 2000

\usepackage{authoraftertitle}
\usepackage{fancyhdr}
% Clear the header and footer
\fancyhead{}
\fancyfoot{}
%\lhead{\includegraphics[height=0.7cm]{../logo/nyp-logo-int.png} }
\rhead{\scriptsize \MyTitle}
%\lfoot{\scriptsize Specialist Diploma in Business \& Big Data Analytics\\ Copyright
%  \copyright\ Nanyang Polytechnic. All Rights Reserved.}
\rfoot{\thepage}
\pagestyle{fancy}


\renewcommand{\topfraction}{0.95}
\renewcommand{\textfraction}{0.02}
\renewcommand{\floatpagefraction}{0.95}


\bibliographystyle{plain}

%%\textwidth      150mm \textheight     210mm \oddsidemargin  -2mm \evensidemargin -2mm

\renewcommand{\baselinestretch}{0.987}

\setlength{\parskip}{0.0in}


\newcommand{\Nturns}{\, \vdash_{\mbox{\scriptsize lnf}} \,}

\newcommand{\tr}[1]{}
%%\newcommand{\tr}[1]{#1}
%%\newcommand{\nottr}[1]{#1}
\newcommand{\nottr}[1]{}

%\newcommand{\implies}{\supset}
\newcommand{\clabel}[1]{\mbox{(#1)}}
\newcommand{\rat}[1]{\rightarrowtail_{#1}}
\newcommand{\arr}{\rightarrow}
\newcommand{\arrow}{\rightarrow}
\newcommand{\Arr}{\Rightarrow}
\newcommand{\atsign}{@}
\newcommand{\simparrow}[0]{\Longleftrightarrow}
\newcommand{\proparrow}[0]{\Longrightarrow}
\newcommand{\comment}[1]{}
\newcommand{\ignore}[1]{}
%\newcommand{\kl}[1]{{\bf KL:#1}} %%%{\marginpar{\sc kl}{\bf #1}}
\newcommand{\kl}[1]{}
\newcommand{\ms}[1]{{\bf MS:#1}}       %%{\marginpar{\sc ms}{\bf #1}}
%%\newcommand{\jw}[1]{\marginpar{\sc jw}{\bf #1}}
%% \newcommand{\answer}[1]{#1}

\newcommand{\pjs}[1]{}
%%\newcommand{\ms}[1]{}
\newcommand{\jw}[1]{}
%%\newcommand{\kl}[1]{}


\newcommand{\pow}{\^{}}
\newcommand{\venv}{\Delta}
\newcommand{\mleq}{\mbox{\tt leq}}
\newcommand{\mmleq}{\leq}
\newcommand{\mas}{\mbox{\tt as}}
\newcommand{\mfix}{\mu}

\newcommand{\mysection}[1]{\vspace*{-2mm}\section{#1}\vspace*{-1mm}}
\newcommand{\mysubsection}[1]{\vspace*{-1mm}\subsection{#1}\vspace*{-1mm}}

\newcounter{cnt}
\newtheorem{ex}{Example}
%\newenvironment{example}{
%        \begin{ex}\rm}%
%    {\hfill$\Box$\end{ex}}

\newenvironment{nexample}{
        \begin{ex}\rm}%
        {\end{ex}}

%%\newtheorem{rem}{Remark}
%%\renewenvironment{remark}{
%%        \begin{rem}\rm}%
%%    {\hfill$\Box$\end{rem}}

%%\newenvironment{myexample}{
%%        \begin{ex}\rm}%
%%  {\hfill $\Diamond$\end{ex}}
%\newenvironment{example}{
%        \begin{example}\rm}%
%    {\end{example}}

\newenvironment{ttline}{\begin{trivlist}\item \tt}{\end{trivlist}}
\newenvironment{ttprog}{\begin{trivlist}\small\item \tt
        \begin{tabbing}}{\end{tabbing}\end{trivlist}}


\newcommand{\figcode}[1]
        {\begin{figure*}[t]#1
        \end{figure*}}


\title{%ITD353 Massively Parallel Computing for Big Data \\
  Introduction to Scala }
%\author{Kenny Zhuo Ming Lu\\
%  \multicolumn{1}{p{.7\textwidth}}{\centering
%  \emph{School of Information Technology \\ Nanyang Polytechnic \\
%    180 Ang Mo Kio Avenue 8, Singapore 569830}}
%}


\begin{document}
\maketitle \makeatactive
\thispagestyle{fancy}

\lstset{language=Python}

%\begin{abstract}
%\end{abstract}


\section{Learning Outcomes} \label{sec:aims}
\begin{itemize}
 \item Start Scala REPL in Scala application development
 \item Execute and observe Scala programs Scala application
   development
 \item Comprehend all the Scala languages features and the program semantics when reviewing Scala source codes
 \item Develop data transformation scripts using Scala
\end{itemize}




\section{Scala Features}
\begin{enumerate}
\item Scala is an \underline{o~~~~~~~~~~~~~~} oriented and \underline{f~~~~~~~~~~~~~~} language. 
\item Scala is a \underline{~~~~~~~~~~~~~~} typed language.
\end{enumerate}

\section{First Scala Program - Hello World}
\begin{enumerate}
\item Check out the source codes.
  \begin{enumerate}
  \item go to Github and download the zip
  \item Clone it from github.
    \begin{code}
      $ cd ~
      $ mkdir git
      $ cd git 
      $ git clone http://github.com/luzhuomi/learning-scala.git
      $ cd learning-scala/codes
    \end{code}
  \end{enumerate}
\item Examine the script {\tt Script.scala}  in {\tt helloworld}.
\item Execute the script with the following
\begin{code}
$ scala Script.scala
\end{code}
\item Examine the code {\tt Main.scala} in {\tt helloworld}.
\item Compile the code
\begin{code}
$ scalac Main.scala
\end{code}
\item Execute the compiled code
\begin{code}
$ scala Main
\end{code}
\end{enumerate}

\section{Scala REPL}
\begin{enumerate}
\item Start a terminal in Linux, type 
 \begin{code}
  $ scala
  \end{code}
 Note that the \$ sign is the command prompt, you should not include it as
 part of the command.
\item Exit python REPL by typing 
 \begin{code}
 scala> :quit
 \end{code}
 Note that the {\tt scala>} sign is the Scala REPL prompt, you should not include it as
 part of the command.
\end{enumerate}
%$

\section{Variables, Values and Assignment Statement}
In a Scala REPL
\begin{enumerate}
\item Declare a variable with name ``first\_name'' and assign a string
  value as ``robin''.
\item Declare a value with name ``last\_name'' and assign a string
  value as ``Williams''.
\item Update the variable ``first\_name'' to a new string value
  ``Robin''
\item If you were to update the value ``last\_name'' to a new string
  ``Hood'', what will happen?
\end{enumerate}
\comment{
\subsection{Sample Answer}
\begin{code}
scala> var first_name = "robin"
first_name: String = robin

scala> val last_name = "Williams"
last_name: String = Williams

scala> first_name = "Robin"
first_name: String = Robin

scala> last_name = "Hood"
<console>:12: error: reassignment to val
       last_name = "Hood"
                 ^
\end{code}
}
\section{Print Statement}
In a Scala REPL
\begin{enumerate}
\item Print the variable ``first\_name'' and value ``last\_name'' individually
\item Use template, print the following
\begin{code}
Robin William (1951 - 2014) 
\end{code}
You need to make use of the variable ``first\_name'' and value ``last\_name'',
and put 1951 and 2014 into the two additional variables. For instance,
assuming you have defined ``first\_name'' and ``last\_name''.
\begin{code}
val bYear = 1951
val dYear = 2014
println(s"$first_name $last_name ($bYear - $dYear)")
\end{code}

\end{enumerate}
\comment{
\subsection{Sample Answer}
\begin{code}
scala>  val bYear = 1951
bYear: Int = 1951

scala>  val dYear = 2014
dYear: Int = 2014

scala> println(s"$first_name $last_name ($bYear - $dYear)")
Robin Williams (1951 - 2014)
\end{code}
}

\section{If-else}

\begin{enumerate}
\item Type the following code snippet in the Scala REPL and observe the output.
\begin{code}
val i = 1
if (i / 2 >= 0.5) {
  println(s" ${i} / 2 is greater than or equal to  0.5") } 
else { 
   println(s"${i} / 2 is less than 0.5") 
}
\end{code}
\end{enumerate}


\section{List and List operation}

\begin{enumerate}
\item Declare a list of integer {\tt l1} with integers 1, 2, 3 and 4.
\item Declare a second list {\tt l2} whose elements are the odd values of {\tt
    l1} incremented by 1.
\item Find out the head and the tail of {\tt l2}.
\item Reverse {\tt l2}.
\item Concatenate {\tt l1} and {\tt l2}
\item Compute the sum of {\tt l1}
\end{enumerate}

\comment{
\subsection{Sample Answer}
\begin{code}
scala> val l1 = List(1,2,3,4)
l1: List[Int] = List(1, 2, 3, 4)

scala> val l2 = for { x <- l1; if x \% 2 == 1} yield (x + 1)// or
l2: List[Int] = List(2, 4)

scala> val l2a = l1.filter ( x => x \% 2 == 1 ).map( x => x + 1)
l2a: List[Int] = List(2, 4)

scala> l2.head
res0: Int = 2

scala> l2.tail
res1: List[Int] = List(4)

scala> l2.reverse
res2: List[Int] = List(4, 2)

scala> l1 ++ l2
res3: List[Int] = List(1, 2, 3, 4, 2, 4)

scala> l1.foldLeft(0)( (s,x) => s + x )
res4: Int = 10
\end{code}
}

\section{Object Oriented Programming}

\begin{enumerate}
\item In the terminal, change the working directory to {\tt
    {}~/git/learning-scala/codes/oop}.
\item Examine the code {\tt OOP.scala}, are you able to identify the
  class constructors, member fields, member methods? Are you able to
  identify the class inheritence? 
\begin{code}
class Person(n:String,i:String) {
	private val name:String = n
	private val id:String   = i
	def getName():String = name
	def getId():String = id
}

trait NightOwl {
	def stayUpLate():Unit
}

class Student(n:String, i:String, g:Double) extends Person(n,i) with NightOwl {
	private var gpa = g
	def getGPA() = gpa
	def setGPA(g:Double) =
	{
		gpa = g
	}
	override def stayUpLate():Unit =
	{
		println("woohoo")
	}
}

class Staff(n:String, i:String, sal:Double) extends Person(n,i) {
	private var salary = sal
	def getSalary() = salary
	def setSalary(sal:Double) =
	{
		salary = sal
	}
}
\end{code}
\item Load the class in the Scala REPL and test it out
\begin{code}
scala> :load OOP.scala
Loading OOP.scala...
defined class Person
defined trait NightOwl
defined class Student
defined class Staff

scala> val tom = new Student("Tom", "X1235", 4.0)
tom: Student = Student@601c1dfc

scala> val jerry = new Staff("Jerry", "T0001", 500000.0)
jerry: Staff = Staff@650fbe32

scala> tom.stayUpLate
woohoo
\end{code}
\end{enumerate}

\section{Functional Programming in Scala}
\begin{enumerate}
\item In the terminal, change the working directory to {\tt
    {}~/git/learning-scala/codes/fp}.
\item Examine the code {\tt Exp.scala}, are you able to identify the
  sealed trait, the case class, and the pattern matching?
\begin{code}
sealed trait Exp
case class Val(v:Int) extends Exp
case class Plus(e1:Exp, e2:Exp) extends Exp

def simp(e:Exp):Exp = e match
{
	case Val(v) => e
	case Plus(Val(0), e2) => e2
	case Plus(e1,e2) => Plus(simp(e1), simp(e2))
}
\end{code}

\item Run it with Scala REPL
\begin{code}
$ scala
scala> :load Exp.scala
scala> val e = Plus(Val(0), Plus(Val(1), Val(2)))
e: Plus = Plus(Val(0),Plus(Val(1),Val(2)))

scala> simp(e)
res0: Exp = Plus(Val(1),Val(2))
\end{code}
%$
\item Note that $x - 0 = x$, $x * 1 = x$, $x / 1 = x$ for all $x$, can
  we extend our {\tt Exp} data type and the simplification {\tt simp} to handle minus, multiplication, and division?
\item Note that the simplification is not througout, e.g. 
\begin{code}
scala> val e2 = Plus(Val(0), Plus(Val(0),Val(2)))
e2: Plus = Plus(Val(0),Plus(Val(0),Val(2)))

scala> simp(e2)
res1: Exp = Plus(Val(0),Val(2))
\end{code}
How can we fix it?
\end{enumerate}

\comment{
\subsection{Sample Answer}

\begin{code}
sealed trait Exp
case class Val(v:Int) extends Exp
case class Plus(e1:Exp, e2:Exp) extends Exp
case class Minus(e1:Exp, e2:Exp) extends Exp
case class Mult(e1:Exp, e2:Exp) extends Exp
case class Div(e1:Exp, e2:Exp) extends Exp

def simp(e:Exp):Exp = e match
{
        case Val(v)            => e
        case Plus(Val(0), e2)  => e2
        case Plus(e1, Val(0))  => e1
        case Plus(e1,e2)       => Plus(simp(e1), simp(e2))
        case Minus(e1, Val(0)) => e1
        case Minus(e1, e2)     => Minus(simp(e1), simp(e2))
        case Mult(Val(1), e2)  => e2
        case Mult(e1, Val(1))  => e1
        case Mult(e1, e2)      => Mult(simp(e1), simp(e2))
        case Div(e1, Val(1))   => e1
        case Div(e1, e2)       => Div(simp(e1), simp(e2))
}

def simpFix(e:Exp):Exp =
{
        val e2 = simp(e)
        if (e == e2)
        {
                e
        }
        else
        {
                simpFix(e2)
        }
}
\end{code}
}


%\item Implement An Interpreter of Lambda Calculus 


\end{document}
